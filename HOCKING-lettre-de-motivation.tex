\documentclass{article}

\usepackage[utf8]{inputenc}
\usepackage[T1]{fontenc}

\usepackage{times}
\usepackage{hyperref}
\usepackage[cm]{fullpage}
\usepackage{fancyhdr}
\usepackage{graphicx}
\usepackage{natbib}

\setlength{\headheight}{9.2pt}
\setlength{\headsep}{5.2pt}
\pagestyle{fancyplain}
\pagenumbering{gobble}
\lhead{\textbf{{\large Toby Dylan Hocking}  \\ \texttt{toby.hocking@mail.mcgill.ca} }}
\chead{{\LARGE \bf 
%Cover Letter
%Letter of Interest
Lettre de Motivation
} }
\rhead{\includegraphics[height=0.75cm]{logo-mcgill}}

\begin{document}

\mbox{ }

Chers collègues dans le comité de sélection,

Je fais de la recherche sur les nouveaux algorithmes statistiques pour
les grands jeux de données. Je suis intéressé à devenir le prochain
professeur dans le département de mathématiques de l'Université de
Québec à Montréal.

Dans la recherche, mes expertises sont l'apprentissage automatique, la
statistique computationnelle, et la bioinformatique. J'ai une
contribution forte à la litterature d'apprentissage automatique, avec
quatre publications dans les conférences principales (NIPS et
ICML). J'ai aussi quelques publications dans les journaux de
statistiques et de bioinformatique. Enfin j'ai corédigé plusieurs
articles dans les journaux biomédicaux, sur des applications de mes
algorithmes dans l'analyse du cancer. Je serai excité d'écrire des
demandes de subvention pour financer des futurs projets de recherche
sur ces sujets. Pour en savoir plus, voir mon plan de recherche
en pièce jointe.
 
J'ai plusieurs expériences avec l'enseignement, qui comprennent des cours
à l'université, des tutoriels lors des conférences internationales, et
l'encadrement des étudiants pour les projets de recherche et
programmation. Je suis excité de poursuivre l'enseignement et l'encadrement. Pour en savoir plus, voir mon plan d'enseignement en pièce jointe.

Même si je ne suis pas francophone d'origine, j'ai fait mon doctorat à
Paris, et je suis courant en français depuis environ 2009. J'ai passé
le test de français « TEF Canada » en mars 2017. J'ai déjà enseigné un
cours sur R en français à l'Université de Montréal, et je serai ravi
d'enseigner d'autres cours en français à l'UQAM.

Pour mes lettres de recommendation je vous conseille de contacter mon
professeur de postdoc actuel, et mes collaborateurs actuels. Vous
pouvez également contacter mes professeurs de doctorat si vous voulez
(mais SVP notez que nous ne travaillons plus ensemble depuis la fin de
ma thèse en 2012).

Merci cordialement d'avoir considéré mon profil, 

Toby Dylan Hocking

\end{document}
