\documentclass{article}

\usepackage{times}
\usepackage{hyperref}
\usepackage[cm]{fullpage}
\usepackage{fancyhdr}
\usepackage{graphicx}
\usepackage{natbib}

\setlength{\headheight}{9.2pt}
\setlength{\headsep}{5.2pt}
\pagestyle{fancyplain}
\pagenumbering{gobble}
\lhead{\textbf{{\large Toby Dylan Hocking}  \\ \texttt{toby.hocking@mail.mcgill.ca} }}
\chead{{\LARGE \bf Teaching Statement} }
\rhead{\includegraphics[height=0.75cm]{logo-mcgill}}

\begin{document}

 \mbox{ }
% \vspace{ -0.5cm}

%\section*{\centering Teaching Statement}

\paragraph{Résumé.} J'aime enseigner des sujets tels que
l'apprentissage automatique, la statistique, la visualisation des
données, la programmation pour l'analyse des données et la
bioinformatique. Je pourrais aussi enseigner des cours sur les
mathématiques, la probabilité et l'informatique. Par exemple, je serai
heureux d'enseigner des cours de premier cycle qui sont pertinentes à
l'apprentissage automatique, comme l'algèbre linéaire, l'analyse
réelle, la probabilité, la statistique, les algorithmes et les
structures de données. Je pourrais aussi enseigner sur des sujets plus
avancés tels que l'optimisation convexe. Dans toutes les cours
j'utiliserais des stratégies de ma philosophie d'enseignement telles
que les sondages, s'adapter au public, utiliser des exemples motivants
et demander au public des questions. Je mettrai l'accent sur
l'évaluation des étudiants par les projets créatifs (résultats
présentés sous la forme d'un rapport de recherche ou d'une affiche)
plutôt que les examens, si possible. Dans les projets de recherche,
j'encouragerai mes étudiants à publier des articles dans les
conférences réspectées et des journaux, et de publier des logiciels
libres sur Internet.

\section{Plans de cours}

\paragraph{Apprentissage automatique.} 
Mon plan pour un cours sur l'apprentissage automatique sera d'utiliser
un tableau et des diapositives pour présenter les mathématiques avec
des dessins géométriques qui aident à comprendre les maths. Je suivrai
la présentation utilisé dans \emph{Elements of Statistical Learning}
par Hastie, \emph{et al}. Je commencerai par des exemples de
problèmes d'apprentissage supervisé, puis présenter des modèles
linéaires de régression et de classification avant de discuter de
principes plus généraux tels que l'optimisation, l'approximation des
fonctions, la validation croisée, le surapprentissage, etc. Enfin, je
vais discuter de plusieurs méthodes non linéaires (par exemple arbres,
boosting, forêts aléatoires, réseaux de neurones), le principe de
relaxation convexe (par exemple des modèles linéaires régularisés,
separateur à vaste marge), et des algorithmes d'apprentissage non
supervisés.

% \paragraph{Plan for class about biostatistics and/or machine learning for
%   bioinformatics.}
% I could also teach a class about applications of biostatistics and/or
% machine learning in bioinformatics. I will use a less technical
% approach, as in \emph{Introduction to Statistical Learning} by James,
% \emph{et al}. I will focus on interpretable methods for regression and
% classification (linear models, decision trees), evaluating models
% (cross-validation), clustering (k-means, mixture models, hierarchical
% clustering), feature selection (greedy stepwise variable selection,
% L1-regularization), analysis of survival data (accelerated failure
% time and Cox models), genomic segmentation (dynamic programming,
% HMMs), and model selection using penalty functions (information
% criteria such as AIC, BIC).


\paragraph{La programmation pour l'analyse de données.} 
Je vais utilisez le livre \emph{Introduction to Data Technologies}
de Paul Murrell pour structurer un cours sur la programmation pour
l'analyse des données. Le livre discute non seulement R mais aussi
SQL, HTML, expressions régulières, et d'autres technologies. Je vais
quand même utiliser la plupart de temps pour discuter la programmation
avec R, et je vais expliquer l'écriture d'un paquetage R. Je vais
discuter des avantages et des inconvénients du CRAN par rapport à
GitHub. Pour une classe avancée je ferai quelques conférences sur les
forces relatives de C, C++ et Python.

\paragraph{La visualisation de données.} 
J'ai enseigné un tutoriel avancé sur la visualisation interactive des
données à useR2016, la conférence internationale sur R. Dans un cours
plus longue je vais me concentrer sur la visualisation des jeux de
données réels en utilisant des logiciels libres (R + ggplot2 et
JavaScript + D3). Je commencerai par montrer quelques exemples
motivants de jeux de données et visualisations informatives. Je vais
mettre l'accent sur le `` Grammar of Graphics '' de Wilkinson, et
comment il peut être utilisé dans la pratique (le paquetage ggplot2
dans R). J'inclurai également une discussion sur la recherche de la
perception optimale des visualisations de données (par exemple les
palettes de couleur de Cynthia Brewer). Enfin, je vais expliquer
comment créer des visualisations interactives en utilisant plusieurs
paquets R, et JavaScript avec D3.

\section{Résumé des expériences d'enseignement} 

% \paragraph{Universit\'e de Qu\'ebec \`A Montr\'eal, Advanced R, Fall
%   2017.} I plan to teach a 15-hour class about advanced topics in R
% (advanced data manipulation and visualization, machine learning, 
% package development, compiled code).

\paragraph{Congrès useR, Tutoriel sur la détection de ruptures, été
  2017.} Avec Rebecca Killick, nous avons créé un cours de 3 heures.

\paragraph{Universit\'e de Montr\'eal, Introduction à R, printemps
  2017.} J'ai enseigné deux cours de 3 heures et quelques exercises
pratiques pour l'apprentissage de R (structures de donnée, entrées/sorties,
manipulation de données, visualization de données).

\paragraph{Congrès useR, Tutoriel sur la visualisation interactive de données, été 2016.} Avec Claus Thorn Ekstr\o m,
nous avons créé un cours de 3 heures.

\paragraph{Ecole des Mines de Paris, Apprentissage automatique,
  printemps 2011.} J'ai créé des exercises pour trois sessions de
travaux pratiques (separateur à vaste marge, modèles graphiques
probabalistes, et clustering).

\section{Expérience en mentorat}

\paragraph{Projets de recherche.}
\begin{itemize}
\item Alexandre Drouin, doctorant, Universit\'e Laval, papier
  d'apprentissage automatique : Max Margin Interval Trees (accepted at
  \emph{NIPS'17}).
\item Carson Sievert, doctorant, Iowa State University, papier de
  statistiques : des nouveaux mots-clés pour l'intéractivité dans le
  grammaire des graphiques (soumis à \emph{Journal of Computational
    and Graphical Statistics}).
\item David Venuto, étudiant de bachelor, McGill University, papier
  d'apprentissage automatique sur un algorithme SVM pour le classement
  (à soumettre à \emph{Journal of Machine Learning Research}).
\item Najmeh Alirezaie, doctorant, McGill University, papier sur
  l'utilisation d'apprentissage automatique pour la pathogenicité des
  variations génétiques (en cours).
\end{itemize}

\paragraph{Projets de programmation dans Google Summer of Code.} J'ai
encadré chaque étudiant lors d'un stage de 3 mois (par skype / courriel).
\begin{itemize}
\item Marlin Na, 2017, navigateur de génome interactif : TnT.
\item Rover Van, 2017, régression censurée : iregnet.
\item Abhishek Shrivastava, 2016, étiquetage et l'apprentissage automatique pour les données génomiques : SegAnnDB.
\item Faizan Khan, 2016--2017, grammaire de graphiques intéractif : animint.
\item Anuj Khare, 2016, régression censurée : iregnet.
\item Qin Wenfeng, 2016, expressions régulières.
\item Akash Tandon, 2016, tests de vitesse pour les paquetages R : Rperform.
\item Ishmael Belghazi, 2015, algorithme de descente de gradient
  stochastique moyenné (Stochastic Average Gradient) : bigoptim.
\item Kevin Ferris, 2015, grammaire de graphiques intéractif : animint.
\item Tony Tsai, 2015, grammaire de graphiques intéractif : animint.
\item Carson Sievert, 2014, grammaire de graphiques intéractif : animint.
\item Susan VanderPlas, 2013, grammaire de graphiques intéractif : animint.
\end{itemize}



\end{document}
