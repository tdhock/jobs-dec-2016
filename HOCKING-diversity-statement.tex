\documentclass{article}

\usepackage{times}
\usepackage{hyperref}
\usepackage[cm]{fullpage}
\usepackage{fancyhdr}
\usepackage{graphicx}

\setlength{\headheight}{9.2pt}
\setlength{\headsep}{5.2pt}
\pagestyle{fancyplain}
\pagenumbering{gobble}
\lhead{\textbf{{\large Toby Dylan Hocking}  \\ \texttt{toby.hocking@mail.mcgill.ca} }}
\chead{{\LARGE \bf Diversity Statement} }
\rhead{\includegraphics[height=0.75cm]{logo-mcgill}}

\begin{document}

\mbox{ }
% \vspace{ -0.5cm}

%\section*{\centering Diversity Statement}

I am committed to encouraging a diverse and inclusive world that
respects all people, regardless of race, sex, gender, sexual
orientataion, culture, etc. My experiences in the international
academic community will certainly add some diversity to your
department. I have worked in USA, France, Japan, and Canada, and I
speak fluent French in addition to my native English. Below I give two examples about how I
encourage diversity in the academic and free/open-source software
communities. 

\paragraph{French language and culture.}

I am a white male who lived in California until the age of 24, when I
moved to Paris for my graduate studies. I decided to move to Paris so
I could learn about French language and culture at the same time as I
pursued my graduate degrees. Now I am fluent in French language, and
married to a French-speaking Canadian.
I have contributed to diversity in the international research
community by encouraging French as a language of scientific
discourse. English is undoubtedly the current standard international
language of science. However, sometimes it is useful to be open to use
other languages for scientific communication. I think this is
especially true for oral communications.

For example, I am president of the organizing committee for the
bilingual scientific conference ``R in Montreal 2018.'' In Montreal,
the diverse audience includes native French speakers, native English
speakers, and people who are bilingual. To encourage a bilingual
conference, we will ask participants to tell us the language of their
communications, and then we will select content based on quotas (50\%
English, 50\% French).

% Another example comes from my experience organizing R user group
% meetings in Montreal. I posted announcements for these monthly
% meetings in both languages, and I encouraged people to speak in one
% language and create written materials in another.

Also, I have given many presentations of my research in
France to audiences of mostly native French speakers. I had to choose
between speaking in English (and confusing most of the audience), or
presenting in French. I found that when I used French for oral
presentations it was more difficult for me, but it probably resulted
in greater audience comprehension, for two reasons. First, the French
people in the audience did not understand English as well as their
native tongue. Second, and more interestingly, speaking French forced
me to slow down and carefully formulate my speech. Overall I think the
experience of giving scientific lectures in French has also helped me
think more clearly about how I communicate in English. 

\paragraph{The international R Google Summer of Code community.}
The Google Summer of Code is an effort to teach students all over the world about free/open-source software development. I have participated as an administrator and mentor for the R project since 2012.

In practice this means I work together with a culturally diverse group of students and mentors from all over the world. For example, I have mentored students from Canada, China, India, and the United States (Montana and Iowa). It is an enriching experience to mentor students from different cultures.

More generally, as a co-administrator we strive for openness and transparency. For example I set up a wiki page where anyone (regardless of national or university affiliation) can post ideas for potential student projects.

And we created a policy for decision-making that results in a diversity of developed packages. For example, if Google gives us funding for two students, we would allocate them to two different R packages (not two students on one R package).

\end{document}
