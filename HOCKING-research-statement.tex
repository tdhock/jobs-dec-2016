\documentclass{article}

\usepackage{times}
\usepackage[cm]{fullpage}
\usepackage{fancyhdr}
\usepackage{graphicx}
\usepackage{natbib}

%\setlength{\headheight}{9.2pt}
\setlength{\headsep}{5.2pt}
\pagestyle{fancyplain}
\pagenumbering{gobble}
\lhead{\textbf{{\large Toby Dylan Hocking}  \\ \texttt{toby.hocking@mail.mcgill.ca} }}

\rhead{\includegraphics[height=0.75cm]{logo-mcgill}}

\begin{document}

 \mbox{ }
% \vspace{ -0.5cm}

\section*{\centering Research Statement}

During 2009-2012, I completed my PhD degree in applied mathematics
from the Ecole Normale Sup\'erieure de Cachan, France. I studied
machine learning algorithms for biology and medicine, under the
supervision of Dr. Francis Bach (INRIA, ENS) and Dr. Jean-Philippe
Vert (Institut Curie, Mines ParisTech). I am currently continuing my
research about machine learning algorithms for biology, as a
postdoctoral research associate under the supervision of Dr. Guillaume
Bourque, in the McGill University Department of Human Genetics.

With recent advances in technology, more data are being generated and
recorded than ever before in human history. For example, in scientific
fields such as medicine and biology, new sensors and DNA sequencing
machines are being used to generate many high-dimensional data
sets. This has led to a need to develop new statistical models and
machine learning algorithms to understand patterns in these data, i.e.
to generate knowledge about how the human genome influences medical
conditions. I am interested in developing new statistical machine
learning algorithms to address these needs, by collaborating with
domain experts who generate these large data sets.

So far my research has been focused on proposing new supervised and
unsupervised machine learning models for clustering, regression,
changepoint detection, ranking, and classification. My contributions
to the statistics and machine learning literature are algorithms for
efficiently solving these discrete and convex optimization problems,
which typically involve very large data sets. I have developed close
collaborations with domain experts in the fields of medicine and
biology, in order to better understand their problems and propose
appropriate solutions. The following sections summarize my previous
research projects, and my future research plan.

\section{Changepoint detection and regression algorithms for DNA copy
  number data}

During my PhD in Paris, I worked in collaboration with medical doctors
at the Institut Curie on semi-automatic methods for diagnosing
neuroblastoma, a cancer that most frequently occurs in small
children. The doctors wanted to give targeted treatments based on the
number of ``breakpoints'' or abrupt changes in DNA copy number data
from the patient's tumor. Previous unsupervised methods for
changepoint detection in these data provided insufficient accuracy
with respect to the doctors' interpretation of the signal/noise in
these data. 

\paragraph{New labeling method for supervised breakpoint detection.}

In work published in \emph{BMC Bioinformatics}
\citep{HOCKING-breakpoints}, we proposed to re-frame the breakpoint
detection problem in terms of supervised learning. In our framework,
the doctors first labeled a subset of their data with positive regions
that contain breakpoints, and negative regions that contain no
breakpoints. We thus have a set of labeled data sequences, and the
goal is to learn a model that provides accurate breakpoint predictions
on a set of un-labeled data sequences. The main contributions of our
paper were to show that existing algorithms can be (1) trained by
learning parameters via grid search that minimize the number of
incorrect training labels, and (2) compared by computing the number of
incorrect test labels in a held-out data set. We performed a
comparison study which showed that the most accurate existing method
was a Gaussian maximum likelihood changepoint detection model. 
% Each
% data sequence is modeled by a piecewise constant segment mean, and a
% dynamic programming algorithm finds the most likely changepoint
% positions, subject to an L0 sparsity constraint on the number of
% changes. 
In this paper we selected the number of changes using a
simple univariate penalty function that depends on the number of data
points, and in our next paper we investigated generalizing that to
multivariate penalty functions.

\paragraph{Regularized linear regression for supervised penalty
  function learning.} 

In work published at ICML \citep{HOCKING-penalties}, we proposed a
fast and accurate supervised machine learning algorithm that exploits
the structure of the labeled changepoint detection problem. We showed
that learning a penalty function for selecting the number of
changepoints is in fact a regression problem with censored
outputs. For each data sequence in which we would like to detect
changepoints, the input is a feature vector of statistics (number of
data points in the sequence, variance estimates, quantiles, etc), and
the output is an interval of log(penalty) values that achieves the
minimum number of incorrect labels. We showed that a data sequence
with a negative label results in a right-censored output (when the
penalty is too small, there are too many changepoints); a positive
label results in a left-censored output (when the penalty is too
large, there are not enough changepoints). When a data sequence has
both positive and negative labels, the output is interval-censored.
We proposed a margin-based convex loss function that exploits the
structure of the censored outputs. We also proposed an L1 penalty to
encourage a sparse, interpretable penalty function. Our main
contribution was a new convex optimization algorithm that minimizes
the resulting convex objective. We showed that our algorithm learns a
multivariate penalty function, and provides more accurate breakpoint
detection than the previous univariate penalty. I implemented these
algorithms in the penaltyLearning R package, which I presented during
a useR2017 tutorial on optimal changepoint detection algorithms
\citep{change-tutorial}.


\paragraph{Functional pruning algorithm for fast changepoint detection} \citep{fpop}


\paragraph{Collaborations using an interactive web app for supervised breakpoint detection} I have used my SegAnnDB system to perform
supervised breakpoint detection in collaborations with scientists at
Intitute Curie, Paris, France \citep{Chicard}; and Aichi Cancer
Center, Nagoya, Japan \citep{Hocking-Leukemia-2016}.

\section{Peak detection in ChIP-seq data}

\paragraph{Constrained dynamic programming algorithm.} \citep{HOCKING-PeakSeg}.

\paragraph{New labeling method for supervised peak detection.}also
showed that scientists' labels are highly consistent, by observing
similar test error rates when training on labels from the same or a
different scientist.

\section{Other research projects}

\paragraph{Convex clustering} \citep{HOCKING-clusterpath}.

\paragraph{Support vector machines for ranking and comparing}\citep{svmcompare}.

\paragraph{Bayesian model of evolution} \citep{HOCKING-evolution}.

\paragraph{Documentation generation for R} \citep{hocking13:inlinedocs}.

\bibliographystyle{plain}
\bibliography{TDH-refs}

\end{document}
