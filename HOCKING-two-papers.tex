\documentclass{article}

\usepackage{times}
\usepackage{hyperref}
\usepackage[cm]{fullpage}
\usepackage{fancyhdr}
\usepackage{graphicx}
\usepackage{natbib}

\setlength{\headheight}{9.2pt}
\setlength{\headsep}{5.2pt}
\pagestyle{fancyplain}
\pagenumbering{gobble}
\lhead{\textbf{{\large Toby Dylan Hocking}  \\ \texttt{toby.hocking@mail.mcgill.ca} }}
\chead{{\LARGE \bf Two Publications} }
\rhead{\includegraphics[height=0.75cm]{logo-mcgill}}

\begin{document}

\mbox{ }

\section{Max-margin interval regression model for supervised changepoint detection}

\url{http://proceedings.mlr.press/v28/hocking13.pdf}

In work published at \emph{ICML'13} \citep{HOCKING-penalties}, we
proposed a fast and accurate supervised learning algorithm
that exploits the structure of the labeled changepoint detection
problem. We showed that learning a penalty function for selecting the
number of changepoints is in fact a regression problem with censored
outputs. We proposed a margin-based convex loss function that
exploits the structure of the censored outputs, and an L1 penalty for
regularization and sparsity. Our main contribution was a new convex
optimization algorithm that minimizes the resulting convex objective
function. We showed that our algorithm learns a multivariate linear
penalty function, and provides more accurate changepoint detection
than the previous univariate penalty. I implemented these algorithms
in the \texttt{penaltyLearning} R package, which I presented during a
\emph{useR2017} tutorial on optimal changepoint detection algorithms
\citep{change-tutorial}.

\paragraph{My contributions.} 
\begin{itemize}
\item I invented the labeling method and created the benchmark data
  sets which allow use of supervised learning algorithms for
  changepoint detection.
\item I recognized that the problem is equivalent to regression with
  censored outputs, and proposed the corresponding optimization
  problems (sections 3.2 and 3.3).
\item I proposed the convex optimization algorithm, and coded and
  implemented it in the \texttt{penaltyLearning} R package.
\item I performed most of the computational experiments, created all
  of the figures/tables, and wrote most of the text (my co-author
  Guillem Rigaill was responsible for some of the experiments and
  text).
\end{itemize}

\section{Up-down constrained changepoint model for peak detection in ChIP-seq data}

\url{http://proceedings.mlr.press/v37/hocking15.pdf}

In work published at \emph{ICML'15} \citep{HOCKING-PeakSeg}, we
proposed a constrained changepoint detection model for peak detection
in ChIP-seq data. These experiments characterize active and inactive
regions of the genome in different individuals and cell types. In
these data, the main analysis task is classify each position of the
genome as either peaks (with large values) or background noise (with
small values). We observed that the existing optimal changepoint model
could sometimes be interpreted in terms of peaks (after up changes)
and background (after down changes). However, in some data sets this
model consists of several consecutive up changes, which is not
directly interpretable in terms of peaks and background. We thus
proposed an additional constraint that up changes must be followed by
down changes, which results in a more interpretable model. Our main
contribution in this paper was to adapt the classical quadratic time
dynamic programming algorithm to this new constraint. Another
contribution was to demonstrate that we can efficiently learn a
penalty function, using the same censored regression algorithm that we
previously used to predict the number of changes in the unconstrained
changepoint detection model. We showed that this method achieves
state-of-the-art peak detection accuracy in all of the labeled data
sets that we created, thereby proving that a single algorithm can
adapt to peak patterns in different data sets.

\paragraph{My contributions.} 
\begin{itemize}
\item I invented the labeling method and created the benchmark data
  sets which allows use of supervised learning algorithms for
  peak detection.
\item I proposed the up-down constraint on segment means, which
  formalizes peak detection as a constrained optimal changepoint
  detection problem.
\item My co-author Guillem Rigaill provided the initial C
  implementation of the dynamic programming algorithm, which I
  modified and published in the \texttt{PeakSegDP} R package.
\item I performed all of the computational experiments, created all of
  the figures/tables, and wrote most of the text (my co-author Guillem
  Rigaill was responsible for some of the text).
\end{itemize}

\bibliographystyle{plain}
\bibliography{TDH-refs}

\end{document}
