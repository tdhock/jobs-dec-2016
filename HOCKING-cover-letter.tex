\documentclass{article}

\usepackage{times}
\usepackage{hyperref}
\usepackage[cm]{fullpage}
\usepackage{fancyhdr}
\usepackage{graphicx}
\usepackage{natbib}

\setlength{\headheight}{9.2pt}
\setlength{\headsep}{5.2pt}
\pagestyle{fancyplain}
\pagenumbering{gobble}
\lhead{\textbf{{\large Toby Dylan Hocking}  \\ \texttt{toby.hocking@mail.mcgill.ca} }}
\chead{{\LARGE \bf 
Cover Letter
%Statement of Interest
} }
\rhead{\includegraphics[height=0.75cm]{logo-mcgill}}

\begin{document}

\mbox{ }

Dear selection committee,

I am a 
%statistics and 
machine learning researcher working on new statistical methods for
analysis and visualization of big data sets.
%collaborating with 
%cancer
%biologists and medical doctors. 
I am interested to become the next
Assistant Professor of Business Analytics at the State University of
New York at New Paltz.

My past research has been highly multi-disciplinary, including many
collaborations with 
%medical doctors,
%biologists, 
both statistical machine learning researchers
%, and computer scientists
and domain experts from various scientific fields. 
% My
% publications thus fall into two categories: new statistical machine
% learning algorithms, and applications in which we use these machine
% learning methods for a better understanding of 
% %cancer 
% biology and genomics. 
My publication record shows a strong
contribution to the machine learning literature, with four
publications at the top-tier conferences (NIPS and ICML). I also have
several contributions to the statistics 
%and bioinformatics 
literature.
%(two publications in {\it Bioinformatics} and one in {\it BMC Bioinformatics}). 
Finally, I have several application papers in biomedical journals such
as {\it Clinical Cancer Research} which have resulted from
collaborations with cancer biologists. I will be excited to pursue new
research collaborations with domain experts from the business
community. 

For my future research, I look forward to continue developing new
statistical machine learning algorithms for analysis of large 
%biological 
data sets. I think that multi-disciplinary projects are absolutely essential
to move forward the fields of 
%medicine, computational biology,
statistics, computer science, and machine learning. 
In my opinion it
is essential that the statistics and machine learning communities
increase the number
of collaborations with 
experts from other fields,
%medical doctors, 
so that our algorithms can be
used to better solve real-world problems.
%understand disease and public health. 
I look
forward to working on new algorithms for analysis of big data 
in collaboration with experts from 
the business community
%other application domains
(operations research, finance, etc). In particular my contributions to
the optimal changepoint detection literature should have many
interesting applications to time series data from business or finance.

I have written a lot of free/open-source code to support my research
papers, so that my algorithms can be widely used for analyzing
%biological
real
data sets. Most of my contributions are to the R language and
environment for data analysis, and I have published several R packages
on CRAN (inlinedocs, neuroblastoma, directlabels, namedCapture,
PeakError, PeakSegDP, PeakSegJoint, PeakSegOptimal, penaltyLearning,
WeightedROC). I have also used Python to implement SegAnnDB, an interactive
machine learning web application for breakpoint detection in genomic DNA copy
number data -- see a live demo on
\url{http://bioviz.rocq.inria.fr/}. 
% I have also contributed code for
% regular expressions to the pandas library for data analysis in
% Python. 
For a list of my recent coding projects, see my GitHub profile
\url{https://github.com/tdhock}. I look forward to teaching the future
generation of students how to write free/open-source software to support
statistical research papers.

% Because I did my PhD in two machine learning labs in Paris, I speak
% fluent French. In March 2017 I passed the TEF Canada French test for
% speaking and understanding French, and I will be excited to teach
% classes in French. 

% I have experience working 2 years at Sangamo BioSciences, a
% biotechnology company in California. My work experience creating a web
% app for statistical analysis of the huge data sets at the company
% provided me with an important perspective about how software systems
% are used in industry. Although I do not have a bachelors degree in
% Engineering, I do have a double major bachelors degree from UC
% Berkeley in related scientific fields (Statistics and Molecular Cell
% Biology). I also have a Masters in Statistics and a PhD in
% Mathematics. I do not have an engineering permit but I could apply for
% one in the future.

% A major focus of my research is creating benchmark data sets which
% formalize real biomedical problems, in order to render these problems
% more accessible to machine learning researchers (who typically do not
% know much about biology).
% For example, during my Ph.D. at the Institute Curie, I worked in
% collaboration with medical doctors treating patients with
% neuroblastoma, a childhood cancer. The doctors were interested to use
% the breakpoints in DNA copy number profiles for diagnosis. I worked
% with the doctors to create the neuroblastoma R package, which contains
% a labeled data set of DNA copy number profiles. The data set includes
% labels which indicate specific genomic regions with or without
% breakpoints -- the labels can be used to easily compute an accuracy
% score for any breakpoint detection model. Benchmark data sets like
% this are essential so that machine learning researchers can work on
% improving the accuracy of algorithms for biomedical data analysis.

% My research contributions in machine learning are very compatible with
% your Data Science initiative. 
% I
% have published several papers at the major machine learning
% conferences (ICML, NIPS), which are highly competitive (about 20\%
% acceptance rate).
% I have collaborated
% with cancer biolgists on models for predicting clinical outcome based
% on DNA copy number profiles. I also have a lot of experience with high
% throughput sequencing data and time series analysis (in particular
% optimal changepoint detection algorithms).
%Finally, most
%Most of my applications papers are collaborations with biologists and
%medical doctors, which I will be excited to continue. 

% I have
% experience using supercomputer clusters such as Guillimin on Compute
% Canada, and I will be excited to use your new ``big data'' lab to
% analyze huge genomic data sets in the context of research and teaching
% projects.

My teaching contributions include creating course materials for
university courses, international conference tutorials, and informal
seminars for my research groups. I have also mentored several students
in research and coding projects. 
% I am very interested to continue
% mentoring graduate students on research projects about new machine
% learning algorithms.
% for better understanding cancer genomics.
Although teaching classes has not been a
major focus for me in the past, I am committed to make it a
priority in the years to come.

My vision for teaching 
%statistics 
business analytics
includes an emphasis on machine
learning and data visualization methods for analyzing large data
sets. I think that it is absolutely critical to both understand the
mathematics behind statistical models, and to be able to implement
algorithms in practice on computers. I contribute extensively to R,
which I believe is as valuable for teaching as it is for research. I
am excited to develop new course materials with which I will
communicate my enthusiasm about these subjects to a new generation of
%statistics 
students.

My experiences in the international academic community will certainly
add some diversity to your department. I have worked in USA, France,
Japan, and Canada, and I speak fluent French in addition to my
native English. Please note my Ph.D. studies in France took only 3
years (2009-2012), which is typical there.
% In France there are no transcripts for doctoral
% studies, so instead I attach a photocopy of an official letter in
% English, ``Certificate of the Award of a Doctoral Degree.'' I do have
% transcripts for my Bachelors and Masters degrees, if that is
% necessary.  
I have been pursuing postdoctoral research since 2013, so
for my letters of recommendation, I have asked my current postdoc
advisor and collaborators.

% {\bf Guillaume Bourque} is my current postdoc advisor at McGill in Montreal, 2014-present.\\
% Email: guil.bourque@mcgill.ca\\
% Title: Assoc. Professor, Department of Human Genetics, McGill University, and\\
% Director of Bioinformatics, McGill University and Genome Quebec Innovation Center. \\
% Address: 740 Dr Penfield Ave, Room 6103, Montréal (Québec) H3A 1A4, Canada.\\
% Telephone: +1(514) 398-7245

% {\bf Guillem Rigaill} is my collaborator, 2012-present.\\
% Email: guillem.rigaill@evry.inra.fr\\
% Title: lecturer (Charg\'e de Recherche) at University of Evry, France.\\
% Address: Universit\'e d'Evry Val d'Essonne,
% Laboratoire de Math\'ematiques et Mod\'elisation d'Evry (UMR 8071),
% I.B.G.B.I., 23 Bd. de France, 91037 Evry Cedex,
% France.\\
% Telephone: +33 1 60 87 45 18

% {\bf Paul Fearnhead} is my collaborator, 2014-present.\\
% Email: p.fearnhead@lancaster.ac.uk\\
% Title: Professor of Statistics, Lancaster University\\
% Address: Mathematics and Statistics Department,
% Office B23, B - Floor, Fylde College,
% Lancaster University,
% Lancaster, LA1 4YF,
% United Kingdom\\
% Tel: +44 (0)1524 594145

% Finally, I grew up in California and I have visited Nevada several
% times in childhood. I would be interested to move to Reno to
% enjoy the beautiful natural environment.

Thanks in advance for your consideration.

Sincerely,

%\vskip 1cm

Toby Dylan Hocking

\end{document}
