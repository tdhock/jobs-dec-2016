\documentclass{article}

\usepackage{times}
\usepackage{hyperref}
\usepackage[cm]{fullpage}
\usepackage{fancyhdr}
\usepackage{graphicx}
\usepackage{natbib}

\setlength{\headheight}{9.2pt}
\setlength{\headsep}{5.2pt}
\pagestyle{fancyplain}
\pagenumbering{gobble}
\lhead{\textbf{{\large Toby Dylan Hocking}  \\ \texttt{toby.hocking@mail.mcgill.ca} }}
\chead{{\LARGE \bf 
%Cover Letter
Letter of Interest
} }
\rhead{\includegraphics[height=0.75cm]{logo-mcgill}}

\begin{document}

\mbox{ }

Dear selection committee,

I am a 
statistical 
machine learning researcher working on new methods for
analysis and visualization of large data sets.
%collaborating with 
%cancer
%biologists and medical doctors. 
I am interested to become the next
Assistant Professor of Statistics at Bucknell University.

% My publication record shows evidence of both collaborative research with cancer biologists, and research into statistical methodology. 
My past research has been highly multi-disciplinary, including many
collaborations with 
medical doctors,
biologists, 
statisticians,
%both statistical machine learning researchers
and computer scientists.
%and domain experts from various scientific fields. 
My
publications thus fall into two categories: new statistical machine
learning algorithms, and applications in which we use these machine
learning methods for a better understanding of 
%cancer 
biology and genomics. 
My publication record shows a strong
contribution to the machine learning literature, with four
publications at the top-tier conferences (NIPS and ICML). I also have
several contributions to the statistics 
and bioinformatics 
literature.
%(two publications in {\it Bioinformatics} and one in {\it BMC Bioinformatics}). 
Finally, I have several application papers in biomedical journals such
as {\it Clinical Cancer Research} which have resulted from
collaborations with cancer biologists. For details please see my
Research Statement. 

My future research projects will focus on new statistical machine
learning algorithms based on mathematical optimization, for efficiently analyzing
large data sets. 
I strongly believe that
the rate of progress in science can be greatly improved
using supervised machine learning, and I look forward to continuing my
research in this direction. For several more specific ideas for future
research projects, please see my Research Statement.


% For my future research, I look forward to continue developing new
% statistical machine learning algorithms for analysis of large 
% %biological 
% data sets. I think that multi-disciplinary projects are absolutely essential
% to move forward the fields of 
% %medicine, computational biology,
% statistics, computer science, and machine learning. 
% In my opinion it
% is essential that the statistics and machine learning communities
% increase the number
% of collaborations with 
% experts from other fields,
% %medical doctors, 
% so that our algorithms can be
% used to better solve real-world problems.
% %understand disease and public health. 
% I look
% forward to working on new algorithms for analysis of big data 
% in collaboration with experts from 
% the business community
% %other application domains
% (operations research, finance, etc). In particular my contributions to
% the optimal changepoint detection literature should have many
% interesting applications to time series data from business or finance.

% I have written a lot of free/open-source code to support my research
% papers, so that my algorithms can be widely used for analyzing
% %biological
% real
% data sets. Most of my contributions are to the R language and
% environment for data analysis, and I have published several R packages
% on CRAN (inlinedocs, neuroblastoma, directlabels, namedCapture,
% PeakError, PeakSegDP, PeakSegJoint, PeakSegOptimal, penaltyLearning,
% WeightedROC). I have also used Python to implement SegAnnDB, an interactive
% machine learning web application for breakpoint detection in genomic DNA copy
% number data -- see a live demo on
% \url{http://bioviz.rocq.inria.fr/}. 
% % I have also contributed code for
% % regular expressions to the pandas library for data analysis in
% % Python. 
% For a list of my recent coding projects, see my GitHub profile
% \url{https://github.com/tdhock}. I look forward to teaching the future
% generation of students how to write free/open-source software to support
% statistical research papers.

% Because I did my PhD in two machine learning labs in Paris, I speak
% fluent French. In March 2017 I passed the TEF Canada French test for
% speaking and understanding French, and I will be excited to teach
% classes in French. 

% I have experience working 2 years at Sangamo BioSciences, a
% biotechnology company in California. My work experience creating a web
% app for statistical analysis of the huge data sets at the company
% provided me with an important perspective about how software systems
% are used in industry. Although I do not have a bachelors degree in
% Engineering, I do have a double major bachelors degree from UC
% Berkeley in related scientific fields (Statistics and Molecular Cell
% Biology). I also have a Masters in Statistics and a PhD in
% Mathematics. I do not have an engineering permit but I could apply for
% one in the future.

% A major focus of my research is creating benchmark data sets which
% formalize real biomedical problems, in order to render these problems
% more accessible to machine learning researchers (who typically do not
% know much about biology).
% For example, during my Ph.D. at the Institute Curie, I worked in
% collaboration with medical doctors treating patients with
% neuroblastoma, a childhood cancer. The doctors were interested to use
% the breakpoints in DNA copy number profiles for diagnosis. I worked
% with the doctors to create the neuroblastoma R package, which contains
% a labeled data set of DNA copy number profiles. The data set includes
% labels which indicate specific genomic regions with or without
% breakpoints -- the labels can be used to easily compute an accuracy
% score for any breakpoint detection model. Benchmark data sets like
% this are essential so that machine learning researchers can work on
% improving the accuracy of algorithms for biomedical data analysis.

% My research contributions in machine learning are very compatible with
% your Data Science initiative. 
% I
% have published several papers at the major machine learning
% conferences (ICML, NIPS), which are highly competitive (about 20\%
% acceptance rate).
% I have collaborated
% with cancer biolgists on models for predicting clinical outcome based
% on DNA copy number profiles. I also have a lot of experience with high
% throughput sequencing data and time series analysis (in particular
% optimal changepoint detection algorithms).
%Finally, most
%Most of my applications papers are collaborations with biologists and
%medical doctors, which I will be excited to continue. 

% I have
% experience using supercomputer clusters such as Guillimin on Compute
% Canada, and I will be excited to use your new ``big data'' lab to
% analyze huge genomic data sets in the context of research and teaching
% projects.

My teaching contributions include creating course materials for
university courses, international conference tutorials, and informal
seminars for my research groups. In the future I will be happy to
create course materials and teach classes that communicate my
enthusiasm for the beautiful mathematics behind statistical machine
learning. Please
% see below for several
% paragraphs explaining my general vision for statistics education in a liberal
% arts setting, and 
see my teaching statement for details about the
classes I would like to teach. I have also mentored several students
in research and coding projects. I will be excited to continue
supervising graduate students on research projects about new
statistical models and algorithms for analyzing large 
%genomic 
data
sets. 
% I am very interested to continue
% mentoring graduate students on research projects about new machine
% learning algorithms.
% for better understanding cancer genomics.
% Although teaching classes has not been a
% major focus for me in the past, I am committed to make it a
% priority in the years to come.

% My vision for teaching 
% statistics 
% %business analytics
% includes an emphasis on machine
% learning and data visualization methods for analyzing large data
% sets. I think that it is absolutely critical to both understand the
% mathematics behind statistical models, and to be able to implement
% algorithms in practice on computers. I contribute extensively to R,
% which I believe is as valuable for teaching as it is for research. I
% am excited to develop new course materials with which I will
% communicate my enthusiasm about these subjects to a new generation of
% %statistics 
% students.

I am committed to encouraging a diverse and inclusive scholarly
environment that respects all people, regardless of race, sex, gender,
sexual orientataion, culture, etc. I look forward to playing a
leadership role in helping to shape and expand diversity initiatives
(e.g. encouraging women to submit papers to machine learning
conferences, which are currently male-dominated). My experiences in
the international academic community will certainly add some cultural
diversity to your department. I have worked in USA, France, Japan, and
Canada, and I speak fluent French in addition to my native English.
%For details please see my Diversity Statement.
% Below I give two
% examples about how I encourage diversity in the academic and
% free/open-source software communities.

Please note my Ph.D. studies in France took only 3
years (2009-2012), which is typical there.
% In France there are no transcripts for doctoral
% studies, so instead I attach a photocopy of an official letter in
% English, ``Certificate of the Award of a Doctoral Degree.'' I do have
% transcripts for my Bachelors and Masters degrees, if that is
% necessary.  
I have been pursuing postdoctoral research since 2013, so
for my letters of recommendation, I have asked my current postdoc
advisor and collaborators.

Thanks in advance for your consideration.

Sincerely,

%\vskip 1cm

Toby Dylan Hocking


% \section*{Diversity Statement}


% \paragraph{French language and culture.}
% I am a white male who lived in California until the age of 24, when I
% moved to Paris for my graduate studies. I decided to move to Paris so
% I could learn about French language and culture at the same time as I
% pursued my graduate degrees. Now I am fluent in French language, and
% married to a French-speaking Canadian.
% I have contributed to diversity in the international research
% community by encouraging French as a language of scientific
% discourse. English is undoubtedly the current standard international
% language of science. However, sometimes it is useful to be open to use
% other languages for scientific communication. I think this is
% especially true for oral communications.

% For example, I am president of the organizing committee for the
% bilingual scientific conference ``R in Montreal 2018.'' In Montreal,
% the diverse audience includes native French speakers, native English
% speakers, and people who are bilingual. To encourage a bilingual
% conference, we will ask participants to tell us the language of their
% communications, and then we will select content based on quotas (50\%
% English, 50\% French).

% % Another example comes from my experience organizing R user group
% % meetings in Montreal. I posted announcements for these monthly
% % meetings in both languages, and I encouraged people to speak in one
% % language and create written materials in another.

% Also, I have given many presentations of my research in
% France to audiences of mostly native French speakers. I had to choose
% between speaking in English (and confusing most of the audience), or
% presenting in French. I found that when I used French for oral
% presentations it was more difficult for me, but it probably resulted
% in greater audience comprehension, for two reasons. First, the French
% people in the audience did not understand English as well as their
% native tongue. Second, and more interestingly, speaking French forced
% me to slow down and carefully formulate my speech. Overall I think the
% experience of giving scientific lectures in French has also helped me
% think more clearly about how I communicate in English. 

% \paragraph{The international R Google Summer of Code community.}
% The Google Summer of Code is an effort to teach students all over the world about free/open-source software development. I have participated as an administrator and mentor for the R project since 2012.

% In practice this means I work together with a culturally diverse group of students and mentors from all over the world. For example, I have mentored students from Canada, China, India, and the United States (Montana and Iowa). It is an enriching experience to mentor students from different cultures.

% More generally, as a co-administrator we strive for openness and transparency. For example I set up a wiki page where anyone (regardless of national or university affiliation) can post ideas for potential student projects.

% And we created a policy for decision-making that results in a diversity of developed packages. For example, if Google gives us funding for two students, we would allocate them to two different R packages (not two students on one R package).

% \section*{Vision for Statistics teaching and scholarship in the liberal arts}
% My vision for teaching statistics in a liberal arts setting includes
% an emphasis on (1) modern computational methods including machine
% learning and data visualization; (2) hands-on workshops where students
% learn to analyze large data sets using free/open-source software; and
% (3) mentoring students on research projects that result in scientific
% papers in major journals and conferences.

% In recent years, the practice of statistics has been transformed by
% the development of fast computers and large data sets. I believe that
% a complete statistics education must now include an emphasis on
% computational methods such as optimization, machine learning and data
% visualization. Whereas in the past one may have performed a simple
% statistical test and the result would be a p-value, today we often
% would like to go further; students will need to know how to perform
% cross-validation, compute predictive models, and visualize there
% results in an informative matter. I am excited to develop new course
% materials with which I will communicate my enthusiasm about these
% subjects to a new generation of statistics students.

% I think that it is absolutely critical to both understand the
% mathematics behind statistical models, and to be able to implement
% algorithms in practice on computers. I contribute extensively to R,
% which I believe is as valuable for teaching as it is for research. I
% believe an introduction to R class should be a prerequisite to all
% other statistics classes, so that R can then be used for interactive
% workshops that accompany all other classes. Interactive
% experimentation on computers is essential to understand mathematical
% models such as regression and classification that are otherwise very
% abstract. For an advanced class on statistical computing, I will also
% explain the advantages and disadvantages of other progamming languages
% (C/C++/Python/JavaScript), and explain how to create R packages.

% Finally I believe that statistics students should have the opportunity
% to pursue research projects, and publish their results in
% internationally recognized journals and conferences. For example I
% will encourage students that I mentor to submit papers to the top
% machine learning conferences, NIPS and ICML. I will also encourage
% them to submit papers to top statistics journals such as {\it Annals of
% Applied Statistics} and {\it Journal of Computational and Graphical
% Statistics}. 

% {\bf Guillaume Bourque} is my current postdoc advisor at McGill in Montreal, 2014-present.\\
% Email: guil.bourque@mcgill.ca\\
% Title: Assoc. Professor, Department of Human Genetics, McGill University, and\\
% Director of Bioinformatics, McGill University and Genome Quebec Innovation Center. \\
% Address: 740 Dr Penfield Ave, Room 6103, Montréal (Québec) H3A 1A4, Canada.\\
% Telephone: +1(514) 398-7245

% {\bf Guillem Rigaill} is my collaborator, 2012-present.\\
% Email: guillem.rigaill@evry.inra.fr\\
% Title: lecturer (Charg\'e de Recherche) at University of Evry, France.\\
% Address: Universit\'e d'Evry Val d'Essonne,
% Laboratoire de Math\'ematiques et Mod\'elisation d'Evry (UMR 8071),
% I.B.G.B.I., 23 Bd. de France, 91037 Evry Cedex,
% France.\\
% Telephone: +33 1 60 87 45 18

% {\bf Paul Fearnhead} is my collaborator, 2014-present.\\
% Email: p.fearnhead@lancaster.ac.uk\\
% Title: Professor of Statistics, Lancaster University\\
% Address: Mathematics and Statistics Department,
% Office B23, B - Floor, Fylde College,
% Lancaster University,
% Lancaster, LA1 4YF,
% United Kingdom\\
% Tel: +44 (0)1524 594145

% Finally, I grew up in California and I have visited Nevada several
% times in childhood. I would be interested to move to Reno to
% enjoy the beautiful natural environment.


\end{document}
