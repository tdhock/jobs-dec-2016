\documentclass{article}

\usepackage{times}
\usepackage{hyperref}
\usepackage[cm]{fullpage}
\usepackage{fancyhdr}
\usepackage{graphicx}
\usepackage{natbib}

\setlength{\headheight}{9.2pt}
\setlength{\headsep}{5.2pt}
\pagestyle{fancyplain}
\pagenumbering{gobble}
\lhead{\textbf{{\large Toby Dylan Hocking}  \\ \texttt{toby.hocking@mail.mcgill.ca} }}
\chead{{\LARGE \bf 
Cover Letter
%Statement of Interest
} }
\rhead{\includegraphics[height=0.75cm]{logo-mcgill}}

\begin{document}

\mbox{ }

Dear selection committee,

I am a 
%statistics and 
%statistical 
machine learning researcher with extensive experience
collaborating with biologists and medical doctors. I am interested to
become the next Assistant Professor of Machine Learning at
Polytechnique Montreal. 

% I have written a lot of free/open-source code to support my research
% papers, so that my algorithms can be widely used for analyzing
% biological data sets. Most of my contributions are to the R language
% and environment for data analysis, and I have published several R
% packages on CRAN (inlinedocs, neuroblastoma, directlabels,
% namedCapture, PeakError, PeakSegDP, PeakSegJoint, PeakSegOptimal,
% penaltyLearning, WeightedROC). I have used Python to implement
% SegAnnDB, an interactive machine learning web application for
% breakpoint detection in DNA copy number data -- see a live demo on
% \url{http://bioviz.rocq.inria.fr/}. I have also contributed code for
% regular expressions to the pandas library for data analysis in
% Python. For a list of my recent projects, see my GitHub profile
% \url{https://github.com/tdhock}.

My publication record shows a strong contribution to the machine
learning literature, with four publications at the top-tier
conferences (NIPS and ICML). My focus in the past has been on convex
and discrete optimization algorithms for shallow models, but I will be
excited to start working on deep learning for future projects. In
particular I am have a lot of experience with dynamic programming
algorithms for optimal changepoint detection models, and I would be
interested to use them as a part of a deep learning algorithm (e.g. to
detect changepoints in the cost function, and adjust learning rates in
consequence). I also have been working on supervised changepoint
detection, a problem for which I think convolutional neural networks
could work quite well (with sufficient labeled data).

My experience includes working with medical doctors and biologists on
applications of machine learning, which is important work that I would
like to continue. I think that interdisciplinary projects are
absolutely essential to move forward the fields of
%medicine, computational biology, 
%statistics 
computer science and machine learning. Most of my research
contributions are in the machine learning literature, and in my
opinion it is essential that we increase the number of collaborations
with medical doctors, so that our algorithms can be used to better
understand disease and public health. In my future reseach I would also
like to collaborate with researchers in other application domains such as
smart cities, natural language processing, computer vision, and
autonomous vehicles.

Because I did my PhD in two machine learning labs in Paris, I speak
fluent French. In March 2017 I passed the TEF Canada French test for
speaking and understanding French, and I will be excited to teach
classes in French. 

I have experience working 2 years at Sangamo BioSciences, a
biotechnology company in California. My work experience creating a web
app for statistical analysis of the huge data sets at the company
provided me with an important perspective about how software systems
are used in industry. Although I do not have a bachelors degree in
Engineering, I do have a double major bachelors degree from UC
Berkeley in related scientific fields (Statistics and Molecular Cell
Biology). I also have a Masters in Statistics and a PhD in
Mathematics. I do not have an engineering permit but I could apply for
one in the future.

% A major focus of my research is creating benchmark data sets which
% formalize real biomedical problems, in order to render these problems
% more accessible to machine learning researchers (who typically do not
% know much about biology).
% For example, during my Ph.D. at the Institute Curie, I worked in
% collaboration with medical doctors treating patients with
% neuroblastoma, a childhood cancer. The doctors were interested to use
% the breakpoints in DNA copy number profiles for diagnosis. I worked
% with the doctors to create the neuroblastoma R package, which contains
% a labeled data set of DNA copy number profiles. The data set includes
% labels which indicate specific genomic regions with or without
% breakpoints -- the labels can be used to easily compute an accuracy
% score for any breakpoint detection model. Benchmark data sets like
% this are essential so that machine learning researchers can work on
% improving the accuracy of algorithms for biomedical data analysis.

% My research contributions in machine learning are very compatible with
% your Data Science initiative. 
% I
% have published several papers at the major machine learning
% conferences (ICML, NIPS), which are highly competitive (about 20\%
% acceptance rate).
% I have collaborated
% with cancer biolgists on models for predicting clinical outcome based
% on DNA copy number profiles. I also have a lot of experience with high
% throughput sequencing data and time series analysis (in particular
% optimal changepoint detection algorithms).
%Finally, most
%Most of my applications papers are collaborations with biologists and
%medical doctors, which I will be excited to continue. 

% I have
% experience using supercomputer clusters such as Guillimin on Compute
% Canada, and I will be excited to use your new ``big data'' lab to
% analyze huge genomic data sets in the context of research and teaching
% projects.

% My experiences in the international academic community will certainly
% add some diversity to your department. I have worked in France, Japan,
% and Canada, and I speak fluent French. Please note my Ph.D. studies in
% France took only 3 years (2009-2012), which is typical there.
% In France there are no transcripts for doctoral studies, so instead I
% attach a photocopy of an official letter in English, ``Certificate of
% the Award of a Doctoral Degree.'' I do have transcripts for my
% Bachelors and Masters degrees, if that is necessary. 
I have been pursuing postdoctoral research since 2013, so for my
letters of recommendation, I have asked my current postdoc advisor and
collaborators.

My teaching contributions include creating course materials for
university courses, international conference tutorials, and informal
seminars for my research groups. I have also mentored several students
in research and coding projects. Although teaching has not been a
major focus for me in the past, I am committed to make teaching a
priority in the years to come.

% My vision for teaching statistics includes an emphasis on machine
% learning and data visualization methods for analyzing large data
% sets. I think that it is absolutely critical to both understand the
% mathematics behind statistical models, and to be able to implement
% algorithms in practice on computers. I contribute extensively to R,
% which I believe is as valuable for teaching as it is for research. I
% am excited to develop new course materials with which I will
% communicate my enthusiasm about these subjects to a new generation of
% statistics students.

% Finally, I grew up in California and I have visited Arizona several
% times in childhood. I would be interested to move to Flagstaff to
% enjoy the beautiful natural environment.

Thanks in advance for your consideration.

Sincerely,

%\vskip 1cm

Toby Dylan Hocking

\end{document}
