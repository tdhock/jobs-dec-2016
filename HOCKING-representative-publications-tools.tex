\documentclass{article}

\usepackage{times}
\usepackage{hyperref}
\usepackage[cm]{fullpage}
\usepackage{fancyhdr}
\usepackage{graphicx}
\usepackage{natbib}

\setlength{\headheight}{9.2pt}
\setlength{\headsep}{5.2pt}
\pagestyle{fancyplain}
\pagenumbering{gobble}
\lhead{\textbf{{\large Toby Dylan Hocking}  \\ \texttt{toby.hocking@mail.mcgill.ca} }}
\chead{{\LARGE \bf Representative Work} }
\rhead{\includegraphics[height=0.75cm]{logo-mcgill}}

\begin{document}

\mbox{ }

I publish free/open-source software to support each of my publications
in the machine learning, statistics, and bioinformatics
literature. The majority of my software tools are published as R
packages on CRAN and GitHub (with some use of Python, C/C++, and
JavaScript). For example, three of my papers and corresponding software
tools are described below.

\paragraph{SegAnnDB: an interactive web app for supervised changepoint
  detection in DNA copy number data.} In work published in
\emph{Bioinformatics} \citep{hocking-SegAnnDB}, we proposed a web app
that facilitates interactive labeling and supervised machine learning
analysis of DNA copy number data. On the practical side, our system
was novel because it allows genomic scientists to not only view the
data but also label obvious signal and noise patterns. This
functionality is essential for collaborations, because when a domain
expert sees an incorrect prediction, he/she can immediately correct it
by adding appropriate labels. Our main statistical contribution was an
efficient dynamic programming algorithm for computing the most likely
K changepoints for a given set of K positive labels. This algorithm
was essential in order to provide a model which fits any set of labels
provided by an expert user. 

Live demo: \url{http://bioviz.rocq.inria.fr/}

Source code: \url{https://github.com/tdhock/SegAnnDB}

\paragraph{Max-margin interval regression model for supervised
  changepoint detection.}
In work published at \emph{ICML'13} \citep{HOCKING-penalties}, we
proposed a fast and accurate supervised learning algorithm
that exploits the structure of the labeled changepoint detection
problem. We showed that learning a penalty function for selecting the
number of changepoints is in fact a regression problem with censored
outputs. We proposed a margin-based convex loss function that
exploits the structure of the censored outputs, and an L1 penalty for
regularization and sparsity. Our main contribution was a new convex
optimization algorithm that minimizes the resulting convex objective
function. We showed that our algorithm learns a multivariate linear
penalty function, and provides more accurate changepoint detection
than the previous univariate penalty. I implemented these algorithms
in the \texttt{penaltyLearning} R package, which I presented during a
\emph{useR2017} tutorial on optimal changepoint detection algorithms
\citep{change-tutorial}.

R package on CRAN: \url{https://cran.r-project.org/package=penaltyLearning}

useR2017 tutorial materials:\\ \url{http://members.cbio.mines-paristech.fr/~thocking/change-tutorial/Supervised.html}

\paragraph{Up-down constrained changepoint model for peak detection in
  ChIP-seq data.}
In work published at \emph{ICML'15} \citep{HOCKING-PeakSeg}, we
proposed a constrained changepoint detection model for peak detection
in ChIP-seq data. We observed that the existing optimal changepoint
model could sometimes be interpreted in terms of peaks (after up
changes) and background (after down changes). However, in some data
sets this model consists of several consecutive up changes, which is
not directly interpretable in terms of peaks and background. We thus
proposed an additional constraint that up changes must be followed by
down changes, which results in a more interpretable model. Our main
contribution in this paper was to adapt the classical quadratic time
dynamic programming algorithm to this new constraint (implementation
in R package PeakSegDP). In more recent work under review at
\emph{Annals of Applied Statistics}
\citep{Hocking-constrained-changepoint-detection}, we proposed a
faster log-linear time algorithm which enables solving the same
problem in much larger data sets (implementation in R package
PeakSegOptimal).

R packages on CRAN: \url{https://cran.r-project.org/package=PeakSegDP}, \\
\url{https://cran.r-project.org/package=PeakSegOptimal}

\bibliographystyle{plain}
\bibliography{TDH-refs}

\end{document}
