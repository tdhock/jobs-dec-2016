\documentclass{article}

\usepackage{times}
\usepackage{hyperref}
\usepackage[cm]{fullpage}
\usepackage{fancyhdr}
\usepackage{graphicx}
\usepackage{natbib}

\setlength{\headheight}{9.2pt}
\setlength{\headsep}{5.2pt}
\pagestyle{fancyplain}
\pagenumbering{gobble}
\lhead{\textbf{{\large Toby Dylan Hocking}  \\ \texttt{toby.hocking@mail.mcgill.ca} }}
\chead{{\LARGE \bf Personal Statement} }
\rhead{\includegraphics[height=0.75cm]{logo-mcgill}}

\begin{document}

\mbox{ }

My vision for teaching statistics in a liberal arts setting includes
an emphasis on (1) modern computational methods including machine
learning and data visualization; (2) hands-on workshops where students
learn to analyze large data sets using free/open-source software; and
(3) mentoring students on research projects that result in scientific
papers in major journals and conferences. 

In recent years, the practice of statistics has been transformed by
the development of fast computers and large data sets. I believe that
a complete statistics education must now include an emphasis on
computational methods such as optimization, machine learning and data
visualization. Whereas in the past one may have performed a simple
statistical test and the result would be a p-value, today we often
would like to go further; students will need to know how to perform
cross-validation, compute predictive models, and visualize there
results in an informative matter. I am excited to develop new course
materials with which I will communicate my enthusiasm about these
subjects to a new generation of statistics students.

I think that it is absolutely critical to both understand the
mathematics behind statistical models, and to be able to implement
algorithms in practice on computers. I contribute extensively to R,
which I believe is as valuable for teaching as it is for research. I
believe an introduction to R class should be a prerequisite to all
other statistics classes, so that R can then be used for interactive
workshops that accompany all other classes. Interactive
experimentation on computers is essential to understand mathematical
models such as regression and classification that are otherwise very
abstract. For an advanced class on statistical computing, I will also
explain the advantages and disadvantages of other progamming languages
(C/C++/Python/JavaScript), and explain how to create R packages.

Finally I believe that statistics students should have the opportunity
to pursue research projects, and publish their results in
internationally recognized journals and conferences. For example I
will encourage students that I mentor to submit papers to the top
machine learning conferences, NIPS and ICML. I will also encourage
them to submit papers to top statistics journals such as Annals of
Applied Statistics and Journal of Computational and Graphical
Statistics. 

\end{document}
