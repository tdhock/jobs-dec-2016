\documentclass{article}

\usepackage{times}
\usepackage{hyperref}
\usepackage[cm]{fullpage}
\usepackage{fancyhdr}
\usepackage{graphicx}
\usepackage{natbib}

\setlength{\headheight}{9.2pt}
\setlength{\headsep}{5.2pt}
\pagestyle{fancyplain}
\pagenumbering{gobble}
\lhead{\textbf{{\large Toby Dylan Hocking}  \\ \texttt{toby.hocking@mail.mcgill.ca} }}
\chead{{\LARGE \bf Teaching Statement} }
\rhead{\includegraphics[height=0.75cm]{logo-mcgill}}

\begin{document}

 \mbox{ }
% \vspace{ -0.5cm}

%\section*{\centering Teaching Statement}

\paragraph{Summary of teaching statement.} I am committed to teaching
classes and mentoring students in research projects. I enjoy teaching
subjects such as machine learning, statistics, data visualization,
programming for data analysis, and bioinformatics. I could also teach
basic classes on the subjects of mathematics, probability, and
computer science. For example I will be happy to teach introductory
undergraduate classes that are relevant to statistical machine
learning, such as linear algebra, real analysis, probability,
statistics, algorithms, and data structures. I could also teach
classes on more advanced topics such as convex and discrete
optimization, which are techniques that I have found useful for
fitting statistical machine learning models to big data sets. In all classes I
will use strategies from my teaching philosophy such as surveys,
adapting to the audience, using motivating examples, and asking quiz
questions. I will emphasize student evaluation via creative coding
projects (results presented as a research report or poster) rather
than examinations, if possible. In research projects I will encourage
my students to publish papers in top-tier peer-reviewed conferences
and journals, and to publish free/open-source code on the internet.

\section{Plans for classes on specific topics}

\paragraph{Plan for statistical machine learning class.} My plan for a
class about statistical machine learning will be to use a blackboard
and slides to present the mathematics alongside intuitive geometric
interpretations, whenever possible. I will follow the presentation
used in \emph{Elements of Statistical Learning} by Hastie, \emph{et
  al}. In particular I will start with examples of supervised learning
problems, then present linear models for regression and classification
before discussing more general principles such as mathematical
optimization, function approximation, cross-validation, overfitting
and underfitting. Finally I will discuss several non-linear methods
(e.g. trees, boosting, random forests, neural networks), the principle
of convex relaxation (e.g. regularized linear models, support vector
machines), and unsupervised learning algorithms.

\paragraph{Plan for class about biostatistics and/or machine learning for
  bioinformatics.}
I could also teach a class about applications of biostatistics and/or
machine learning in bioinformatics. I will use a less technical
approach, as in \emph{Introduction to Statistical Learning} by James,
\emph{et al}. I will focus on interpretable methods for regression and
classification (linear models, decision trees), evaluating models
(cross-validation), clustering (k-means, mixture models, hierarchical
clustering), feature selection (greedy stepwise variable selection,
L1-regularization), analysis of survival data (accelerated failure
time and Cox models), genomic segmentation (dynamic programming,
HMMs), and model selection using penalty functions (information
criteria such as AIC, BIC).

\paragraph{Plan for class about data visualization.} For a class about
data visualization I will focus on how to visualize real data sets
using free/open-source programming tools (R+ggplot2 and
JavaScript+D3). Before mentioning any details about writing code for
graphic design, I will start by showing a few motivating examples of
data sets and informative visualizations. I will emphasize graphic
design using Wilkinson's ``Grammar of Graphics,'' and how it can be
used in practice for non-interactive data visualization (the ggplot2
package in R). I will also include a discussion of research into
methods for optimal perception of data visualizations (e.g. Cynthia
Brewer's color palettes). Finally, I will explain how to create
interactive data visualizations using JavaScript and the Data-Driven
Documents (D3) library.

\paragraph{Plan for class about programming for data analysis.} I will
use Paul Murrell's book \emph{Introduction to Data Technologies} to
structure a class about programming for data analysis. One of its main
strengths is that it covers not just R but also SQL, HTML, regular
expressions, and other technologies. I will spend most of the class
time on R programming, and explain how to write an R package. I will
discuss the advantages and disadvantages of the classic CRAN
distribution system with respect to GitHub. For an advanced class I
will also include a few lectures about the relative strengths of C,
C++, and Python.

\section{Summary of experience teaching classes} 

% \paragraph{Universit\'e de Qu\'ebec \`A Montr\'eal, Advanced R, Fall
%   2017.} I plan to teach a 15-hour class about advanced topics in R
% (advanced data manipulation and visualization, machine learning, 
% package development, compiled code).

\paragraph{International useR conference, Changepoint Detection
  Tutorial, Summer 2017.} With my co-presenter Rebecca Killick, we
created course materials and presented a 3-hour tutorial about R
packages for optimal changepoint detection. 

\paragraph{Universit\'e de Montr\'eal, Introduction to R, Spring
  2017.} I taught two 3-hour lectures and practical sessions about
introductory R programming (data structures, input/output, data
manipulation, data visualization).

\paragraph{International useR conference, Interactive Graphics
  Tutorial, Summer 2016.} With my co-presenter Claus Thorn Ekstr\o m,
we created course materials and presented a 3-hour tutorial about R
packages for interactive graphics.

\paragraph{Mines ParisTech, Machine Learning, Spring 2011.} As
teaching assistant, I taught three lectures and practical sessions
about machine learning algorithms (support vector machines, graphical
models, and clustering).

\section{Summary of experience mentoring students}

\paragraph{Research projects.} I have mentored the following
students in research projects.
\begin{itemize}
\item Alexandre Drouin, PhD student, Universit\'e Laval, machine
  learning paper about a decision tree model for regression with
  censored outputs (accepted at \emph{NIPS'17}).
\item Carson Sievert, PhD student, Iowa State University, statistics
  paper about new keywords for interactivity in the grammar of
  graphics (under review \emph{Journal of Computational and Graphical
    Statistics}).
\item David Venuto, Research Assistant, McGill University, machine
  learning paper about learning ranking functions in data such as
  chess matches (to submit to \emph{Journal of Machine Learning
    Research}).
\item Najmeh Alirezaie, PhD student, McGill University, application
  paper about learning a score for predicting clinical pathogenicity
  of genetic variants (work in progress).
\end{itemize}

\paragraph{Google summer of code projects.} I have mentored the
following students in coding free/open-source software.
\begin{itemize}
\item Marlin Na, 2017, interactive genome browser.
\item Rover Van, 2017, regularized interval regression.
\item Abhishek Shrivastava, 2016, interactive system for labeling and
  machine learning in genomic data.
\item Faizan Khan, 2016--2017, interactive grammar of graphics.
\item Anuj Khare, 2016, regularized interval regression.
\item Qin Wenfeng, 2016, regular expressions.
\item Akash Tandon, 2016, performance testing R packages.
\item Ishmael Belghazi, 2015, stochastic average gradient algorithm.
\item Kevin Ferris, 2015, interactive grammar of graphics.
\item Tony Tsai, 2015, interactive grammar of graphics.
\item Carson Sievert, 2014, interactive grammar of graphics.
\item Susan VanderPlas, 2013, interactive grammar of graphics.
\end{itemize}

\section{Teaching Philosophy}

In general, my teaching philosophy emphasizes three guiding
principles: (1) adapt to the target audience, (2) start with
motivating examples, and (3) frequent interactions with and between
students.

\paragraph{Adapting to the audience.} The best teaching strategy will
depend not only on the subject of the course, but also the target
audience. For example, if I were to teach machine learning to an
audience of biologists, I will avoid too many mathematical details,
and instead prepare slides with examples and figures based on real
data sets from biology. However, for teaching machine learning to
mathematicians or statisticians, I think it will be better to focus on
showing mathematical demonstrations and geometric interpretations of
the models and algorithms.

\paragraph{Motivating examples.} It is important to start any class
with an interesting example problem that students can think about over
the course of the class, and that will be solvable with the tools
presented in the class. For example, in my useR2017 tutorial on
optimal changepoint detection algorithms, I used a data set of labeled
neuroblastoma tumors as a motivating example. As another example, I
taught a class on named capture regular expressions for my current
colleagues who are mainly genomics researchers. The motivating example
that I used in the beginning of the class was converting a genomic
location string to a Python dictionary; after the class, several
students said that it was a very convincing example that motivated
them to learn more about regular expressions.

\paragraph{Frequent interactions.} Finally, I think it is essential to
encourage interactions with and between students. I take frequent
breaks in my lectures to ask if the audience has questions. It is also
very helpful to ask the audience to think about a quiz question, then
to give them 1 minute of discussion with their neighbor to try to find
an answer. For example, in my class about clustering, after asking the
students to compute the log-likelihood of several Gaussian mixture
models, I asked ``does the model with one or two components have a higher
log-likelihood? Why?'' The resulting discussions helped students
understand overfitting and the need for model selection
criteria. After every class, I asked students to fill out an anonymous
survey with questions such as ``would you prefer explanations in
English or French?'' and ``what were the easiest and most difficult
concepts to understand?''


\end{document}
